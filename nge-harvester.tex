\documentclass{webofc}
\usepackage[varg]{txfonts}
\usepackage{natbib}

\usepackage{xcolor}

\usepackage[justification=centering]{caption}

\def\correspondingauthor{\footnote{Corresponding author: \url{matteo.turilli@rutgers.edu}}}

\begin{document}

\title{PanDA and RADICAL-Pilot Integration: Enabling the Pilot Paradigm on HPC Resources}

\author{
  \firstname{Andre} \lastname{Merzky}\inst{2} \and
  \firstname{Pavlo} \lastname{Svirin}\inst{1} \and
  \firstname{Matteo} \lastname{Turilli}\inst{2,}\correspondingauthor
}

\institute{Rutgers University, New Brunswick, NJ, USA \and
           Brookhaven National Laboratory (BNL), Upton, NY, USA
}

\abstract{PanDA executes millions of ATLAS jobs a month on Grid systems with
more than 300,000 cores. Currently, PanDA is compatible only with few
high-performance computing (HPC) resources due to different edge services and
operational policies; does not implement the pilot paradigm on HPC; and does
not dynamically optimize resource allocation among queues. We integrated the
PanDA Harvester service and the RADICAL-Pilot (RP) system to overcome these
limitations and enable the execution of ATLAS, Molecular Dynamics and other
workloads on HPC resources. This paper offer two main contributions: (1)
introducing PanDA Harvester and RADICAL-Pilot, two systems independent
developed to support high-throughput computing (HTC) on high-performance
computing (HPC) infrastructures; (2) describing the integration between these
two systems to produce a middleware component with unique functionalities,
including the concurrent execution of heterogeneous workloads on the Titan
OLCF machine. We integrated Harvester and RP by prototyping a Next Generation
Executor (NGE) to expose RP capabilities and manage the execution of PanDA
workloads. In this way, we minimized the reengineering of the two systems,
allowing their integration while being in production.}

\maketitle

% ---------------------------------------------------------------------------
\section{Introduction}
\label{intro} 

Production ANd Distributed Analysis (PanDA)~\cite{PanDA} is the Workload
Management System (WMS)~\cite{maeno2014evolution} used by the ATLAS
experiment at the Large Hadron Collider (LHC) to execute scientific
applications on widely distributed resources. PanDA is designed to support
the execution of distributed workloads via
pilots~\cite{turilli2018comprehensive}. Pilot-capable WMS enable high
throughput computing (HTC) by executing tasks via multi-level scheduling
while supporting interoperability across multiple sites. Special jobs (i.e.,
pilots) are submitted to each site and, once active, tasks are directly
scheduled to each job for execution without passing through the site's batch
and scheduler systems~\cite{henderson1995job}. This reduces queue time,
increasing task throughput. Pilots are particularly relevant for LHC
experiments, where millions of tasks are executed across multiple sites every
month, analyzing and producing petabytes of data.

The implementation of PanDA WMS consists of several interconnected
subsystems, communicating via dedicated application program interfaces (API)
or HTTP messaging, and implemented by one or more modules. Databases are used
to store stateful entities like tasks, jobs and input/output data, and to
store information about sites, resources, logs, and accounting. PanDA can
execute ATLAS tasks on a variety of infrastructures, including high
performance computing facilities. In the past three years, PanDA was used to
process around 11 million jobs on Titan, a leadership-class machine managed
by the Oak Ridge Leadership Computing Facility (OLCF) at the Oak Ridge
National Laboratory (ORNL).

Ref.~\cite{Doleynik2017high} describes in detail how PanDA has been deployed
on Titan. Importantly, it shows that while PanDA implements the pilot
abstraction to execute tasks on grid resources, on high-performance computing
(HPC) facilities PanDA submits jobs directly to the batch system of the
machine. The size and duration of these jobs depend on resource availability
and workload requirements. PanDA polls resource availability from Titan every
ten minutes via the backfill functionality~\cite{moab-backfill} of the Moab
scheduler~\cite{guide2011moab} and, in case of available resources, submits a
job requiring that amount of resources. The workload of this job is sized so
to allow its execution within the available walltime. PanDA uses Titan’s
resources to execute Geant4 event simulations, a specific type of workload
that is amenable to execute via the machine’s batch
system~\cite{agostinelli2003geant4}.

In this paper, we present the integration between Harvester
(Section~\ref{section:harvester}) and Next Generation Executor
(NGE)~\cite{nge-git} (Section~\ref{sec:rp}). Harvester is a new
resource-facing service developed by ATLAS to enable execution of ATLAS
workloads on HPC machines, bringing further coherence in the PanDA stack
across the support of different types of infrastructures, like grid and HPC.
NGE is a Representational State Transfer REST~\cite{battle2008bridging}
interface developed for
RADICAL-Pilot~\cite{rp-docs,Merzky2018DesignAP,merzky2018using}, a pilot
system developed by the Research in Advanced DIstributed Cyberinfrastructure
and Applications Laboratory (RADICAL)~\cite{radical-web} and designed to
support high-throughput computing on HPC infrastructures, including
leadership-class machines like Titan. Different from Harvester, NGE enables
to submit a pilot job via the batch system of Titan and then directly
schedule tasks on the acquired resources without queuing on the machine batch
system. In this way, tasks can be executed immediately while respecting the
policies of the HPC machine.

PanDA can benefit from pilot capabilities in several ways. Pilot do not
require to package a workload into a batch submission script, simplifying the
deployment requirement. Further, pilots enable the concurrent and sequential
execution of a number of tasks, until the available walltime is exhausted:
Concurrent because a pilot holds multiple nodes of an HPC machine, enabling
the execution of multiple tasks at the same time; sequential because when a
task completes, another task can be executed on the freed resources. In this
way, tasks can be late bound to an active pilot, depending on the current and
remaining availability. This is important because, in principle, ATLAS would
not have to bind a specific portion of tasks to an HPC machine in advance but
it could bind tasks only when the HPC resources become available. This would
bring further coherence to the ATLAS software stack, as pilots and late
binding are already used for grid resources.

RADICAL-Pilot offers a number of distinctive features. Among these, the most
relevant for ATLAS is the possibility to run arbitrary tasks on any given
pilot. Specifically, RADICAL-Pilot decouples the management of a pilot, the
coordination of task executions on that pilot, and what each task executes.
For example, if a task executes a Geant4 simulation and another a molecular
dynamic simulation, both tasks can run at the same time on the same pilot.
Further, RADICAL-Pilot also enables the concurrent execution of different
tasks on Central processing units (CPU) and Graphics Processing Units (GPU),
allowing for the full utilization of HPC worker nodes resources. These
capabilities will enable PanDA to transition from a workload management
system specifically designed to support the execution of ATLAS workloads, to
a system for the execution of general purpose workloads on HPC machines.


% ---------------------------------------------------------------------------
\section{Harvester}
\label{section:harvester}

Harvester~\cite{Megino_2017} is a resource-facing service developed for the
ATLAS experiment at the European Organization for Nuclear Research (CERN)
since 2016 with a wide collaboration of experts. Harvester is stateless with
a modular design to work with different resource types and workloads. The
main objectives of Harvester are: (i) serving as a common machinery for pilot
provisioning on all grid computing resources accessible to ATLAS; (ii)
providing a commonality layer, bringing coherence between grid and HPC
implementations; (iii) implementing capabilities to timely optimize CPU
allocation among various resource types, removing batch-level static
partitioning; and (iv) tightly integrating PanDA and resources to support the
execution of new types of workload.

Figure~\ref{fig:harvester-architecture} shows a schematic view of Harvester's
architecture. Harvester is a stateless service with a local master database
and a central slave database. The local database is used for real-time
bookkeeping close to resources, while the central database is periodically
synchronized with the local database to provide resource information to the
PanDA server. The PanDA server uses this information together with a global
overview of workload distribution to orchestrate a set of Harvester
instances. Communication between Harvester and the PanDA server is
bidirectional: Harvester fetches job descriptions that have to be executed
from PanDA Server using a communicator component, and the same component is
used to report back the status of these jobs.


\begin{figure}
  \centering
  \includegraphics[width=0.30\textwidth]{figures/panda-harvester-overview-color.png}
  \caption{Harvester architecture.}
  \label{fig:harvester-architecture}
\end{figure}

Figure~\ref{fig:harvester-modules} shows how harvester accesses resources
through different plugins. Multiple Harvester instances run on one or more
edge nodes of an HPC center, and access the compute nodes of HPC resources
through local HPC batch systems using specific submission plugins. The status
of each batch submission is tracked using monitor plugin. Input and output
data are transferred with various stager and preparator plugins, which
implement connectors for Rucio~\cite{garonne2014rucio}, Globus
Online~\cite{foster2011globus} and other data transfer clients. Communication
among the components is implemented via the local Harvester database.

\begin{figure}
  \centering
  \includegraphics[width=0.49\textwidth]{figures/panda-harvester-modules.pdf}
  \caption{Modules used for PanDA-NGE interaction.}
  \label{fig:harvester-modules}
\end{figure}

In the context of the BigPanDA project, several instances of Harvester have
been deployed on front nodes of BNL Institutional Cluster, NERSC (Cori),
Titan and Jefferson Lab clusters. These instances are serving experiments
like ATLAS, LQCD, nEDM, IceCube and LSST.


% ---------------------------------------------------------------------------
\section{RADICAL-Pilot and Next Generation Executor}\label{sec:rp}

RADICAL-Pilot (RP) is a pilot system which enables task-level concurrency on
a variety of infrastructures, including HPC and grid systems of the Extreme
Science and Engineering Discovery Environment (XSEDE)~\cite{towns2014xsede},
and the HPC machines Cheyenne at NCAR-Wyoming Supercomputing Center
(NWSC)~\cite{cheyenne}, Blue Waters at the National Center For supercomputing
Applications (NCSA)~\cite{bluewaters}, and Rhea, Titan and Summit at the Oak
Ridge National Laboratory (ORNL)~\cite{olcf-resources}.

As a pilot system, RP allows scheduling jobs on HPC machines (or virtual
machines on cloud infrastructures) to acquire computing
resources~\cite{turilli2018comprehensive}. Once these resources become
available, RP directly schedules tasks on them, i.e., without using the
machine’s batch system. In this way, tasks do not wait in the queue but
execute immediately, enabling high-throughput on HPC machines. It is
important to stress that RP, and pilot systems in general, do not ``game''
the scheduling and security policies of the HPC machines: resource
acquisition is performed via the batch system and resources remain available
only for the given walltime. Jobs wait in the queue alongside all the other
jobs and are subjected to queue, allocation and fair-use policies of the
system. Pilots are own by the same user that submitted the job to the batch
system and RP does not enable pilot multitenancy: only the owner of the job
and therefore of the pilot can schedule tasks on that pilot.
	
Among pilot systems, RP has unique capabilities and architectural properties.
RP offers concurrent execution of heterogeneous tasks on the same pilot,
supporting both CPU and GPU. This means that tasks with diverse requirements,
some using different number of cores and/or work nodes, others using one or
more GPUs, can be executed at the same time on the same pilot, without having
to be queued on the batch system or the HPC resource. Further, RP supports
more than twelve task launching methods---e.g., ssh~\cite{ssh}, mpirun,
aprun, openmpi~\cite{gropp1999using}---and all the major HPC batch
systems---e.g., SLURM~\cite{yoo2003slurm}, PBS~\cite{henderson1995job}, or
LSF~\cite{zhou1992lsf}. In this way, RP can enable pilot capabilities on
machines with different architectures and software environments.

RP isolates the execution of each tasks into a dedicated process, enabling
concurrent execution of heterogeneous tasks by design. For example, RP can
run a bag of 65,000 heterogeneous tasks, requiring between 1 and 384 cores,
some running on CPU cores, others on GPU, some using OpenMP other MPI. These
tasks may run in several ‘generations’, each with varying degree of
concurrency and combination of types of tasks. Further RP can concurrently
manage multiple pilots both on the same machine or submitted across a set of
machines. Tasks can be late-bound to available resources, using different
scheduling algorithms across resources and within each resource.
	
Architecturally, RP is designed following a so called ‘building blocks’
approach to be self-sufficient, interoperable, extensible and partially
composable~\cite{turilli2018building}. Self-sufficient because RP
independently implements the necessary and sufficient set of functionalities
for describing and managing pilot and task entities; interoperable in terms
of type of workload, resource, and execution paradigm; and extensible as new
properties can be added to the pilot, task and resource descriptions, and
more functionalities can be implemented for this entities. Currently,
composability is partially designed and implemented: while the pilot-API can
be used by both users and other systems to describe and execute task-based
workloads, RADICAL-Pilot requires
RADICAL-SAGA~\cite{goodale2006saga,merzky2015saga,radical-saga} to interface
to HPC resources. A prototype interface to cloud resources based on
LibCloud~\cite{LibCloud} is available and a general-purpose resource
connector component is under development.
	
Figure~\ref{fig:rp-arch} shows RP architecture and execution model. RP
assumes a multi-task application written in Python, using RP's Pilot API.
This API let users to describe tasks---called compute units (CU)---and pilots
(Figure~\ref{fig:rp-arch}.1). Each compute unit description has a set of
properties, including the name of the executable launched by RP, the type and
amount of cores this executable will need, whether it requires Message
Passing Interface (MPI)~\cite{gropp1999using}, input files and so on. Pilot
descriptions also have a set of properties, including the endpoint where the
resource request should be submitted, the type and amount of resources to
acquire, and the walltime of these resources. The RP API includes classes to
create both pilot and unit managers to which pilot and unit descriptions are
assigned for execution. Once managed, pilots are submitted to the indicated
endpoint (Figure~\ref{fig:rp-arch}.2), and once scheduled pilot bootstrap an
agent (Figure~\ref{fig:rp-arch}.3). Unit manager(s) schedule compute units
onto available agent(s) (Figure~\ref{fig:rp-arch}.4) and each agent uses a
resource-specific executor to run the units (Figure~\ref{fig:rp-arch}.5).

\begin{figure}
  \centering
  \includegraphics[width=0.75\textwidth]{figures/rp_arch.pdf}
  \caption{RADICAL-Pilot (RP) architecture and execution model. Architecture:
           interfaces (white), modules (purple), components (yellow), pilot
           entity (green), task entity called compute unit (red). Execution
           model: Pilot and units are described via the Pilot API (1); pilots
           are submitted to the indicated resource(s) (2); each pilot
           bootstraps an agent (3); compute units are scheduled onto
           available agent(s) (4) and executed (5).}\label{fig:rp-arch}
\end{figure}
	
Note that compute units are scheduled once the agent is active and that unit
descriptions need to be transferred from the client to the agent. This means
that the latency of each unit transfer reduces the overall utilization of the
available resources as no unit can be executed while in transit. To address
this issue, RP transfers units in bulk, reducing the overall latency of unit
transfer to a single round trip. Further, units are scheduled via a database
instance that needs to be reachable by both the machine on which the unit
managers are instantiated, and the remote machine on which the units will
have to be executed.
	
RP supports the execution of units which require staging of input and/or
output data. Units’ input and output files are managed by stager components
not represented in Figure~\ref{fig:rp-arch}. Data staging is performed
independently of the pilot lifetime, and can thus overlap with the pilot's
queue waiting time, thus reducing the impact on resource utilization.
	
The next generation executor (NGE) is a REST API that enables running RP as a
service. The REST API is a close semantic representation of the underlying
native Pilot API.  The service manages the lifetime of the RP unit and pilot
managers, and of the RP pilot agent, on behalf of a client. As is often the
case when translating a module or library API (which is usually invoked in a
process-local call stack) into a REST API (which is usually called over a
network link), care has to be taken to not let even moderate network
latencies impact the overall API performance.  The NGE API manages to avoid
that problem by adding support for bulk operations for those API methods most
prone to latency impacts: unit submission and state updates.
	
NGE service also provides functionality that extends the scope of the Pilot
API: instead of requiring the client to specify pilot size, runtime and
configuration, it implements policy driven automatisms to shape a pilot based
on the received workload and on resource availability. NGE submits those
auto-shaped pilots to the resource batch queue or it can inspect the target
resource's backfill availability, shaping the pilots so that they fit the
available backfill constraints.
	
NGE's pilot shaping capabilities are policy driven: they can be adapted to
the configuration of the target resource, but also to the specific
requirements of diverse use cases. This makes NGE deployments configurable
for specific user groups and projects. In the context of the presented use
case, NGE is configured to mimic properties which were available in PanDA's
native execution backend, which semantically eases the integration between
the two systems.


% ---------------------------------------------------------------------------
\section{Integrating Harvester and NGE}\label{sec:integration}

We deployed Harvester, NGE, RP, and the MongoDB~\cite{chodorow2013mongodb}
instance needed by RP on three separate containers provided by OLCF via their
OpenShift service~\cite{openshift}. The container used for NGE and RP can
directly submit jobs to the PBS batch system~\cite{henderson1995job} of
Titan, while the MongoDB instance can be reached by both Titan and the NGE
and RP container. This enables submission of pilot jobs to Titan and
scheduling of tasks on the resources of that pilot once scheduled and
bootstrapped.

A separate instance of Harvester has been set up in order to run experiments
of PanDA and NGE interaction. Because of differences in the deployment
environment, Harvester and NGE instances have been installed on different
containers, enabling communication via a secure tunnel. A special module for
job submissions to NGE and job monitoring has been developed for Harvester.
The functionality of these modules have been tested with dummy jobs as well
as with samples of ATLAS and Molecular
Dynamics~\cite{3b6dad414e794d36954333f8f177f47c} workloads.

Figure~\ref{fig:integration} shows details of the coordination protocol
between Harvester and NGE and of the execution model of the integrated
system. RP initiates resource acquisition by submitting a configurable number
of pilots to Titan (Figure~\ref{fig:integration}.1). Once one or more pilots
become available, RP publishes the aggregate number of cores and their time
availability via NGE (Figure~\ref{fig:integration}.2). Note how NGE abstracts
the granularity of pilot resources into a resource overlay described by the
total core availability, partitioned based on the amount of time cores are
available. This partitioning is made necessary by the stacked availability of
multiple pilots on the same resource.

\begin{figure}
  \centering
  \includegraphics[width=0.75\textwidth]{figures/integration.pdf}
  \caption{Integration between Harvester and NGE deployed at OLCF to manage
           the execution of Particle Physics and Molecular Dynamics
           workloads. Harvester, NGE and RP, and MongoDB are deployed on
           containers provided and managed via
           OpenShift.}\label{fig:integration}
\end{figure}

Harvester can poll NGE to know how many cores are available and for how long
they will remain available (Figure~\ref{fig:integration}.3), and push a
number of tasks to NGE for execution (Figure~\ref{fig:integration}.4). Note
that before pushing tasks to NGE, Harvester makes sure that the input data
required by these tasks are available to the work nodes of Titan. This is
done by copying or linking the relevant data to a shared file system, either
local to the cluster or available on the OLCF network
(Figure~\ref{fig:integration}.5). The number of tasks pushed to NGE is
calculated on the base of aggregated task requirements in terms of number of
cores and walltime. Tasks are pushed in bulk so to avoid latency and other
overheads associated with pushing single tasks. Once tasks are pushed into
NGE, these are translated into compute unit descriptions for RP
(Figure~\ref{fig:integration}.6) and then executed on the available pilot
resources (Figure~\ref{fig:integration}.7-9) as described in
Sec.~\ref{sec:rp}. Once executed, RP stages out units' output to a filesystem
accessible by Harvester (Figure~\ref{fig:integration}.10) and Harvester
collects this files, terminating the execution cycle
(Figure~\ref{fig:integration}.11).

NGE can be configured to use RP to submit jobs to Titan in two concurrent or
single modes: standard and backfill. Standard mode supports submission of
jobs to one or more of the five batch queues available on Titan. In this
mode, job submissions are exactly the same as any other job submitted to
Titan for execution by any other user. In backfill mode, RP uses the
\texttt{showbf}~\cite{showbf} command of the Moab scheduler to poll the
current node and walltime availability, creates a pilot description
requesting that availability and submits an equivalent job to the suitable
queue. In this way, RP increases the chances for that job to be immediately
scheduled on Titan, reducing the job (and therefore pilot) queue waiting time
and operating as Harvester but with full-pilot capabilities. This allows for
Harvester to submit multiple generations of tasks for execution on Titan,
maintaining a predefined ‘pressure’ on Titan queues. As a result, the
Havester/NGE integrated system execution model and operational modality is
consistent across Grid and HPC, reducing the differences in how tasks are
mapped between the two types of infrastructure.


% ---------------------------------------------------------------------------
\section{Conclusion and Next Steps}

In this paper we introduced two distinct systems---Harvester and
RADICAL-Pilot---and an interface---Next Generation Executor (NGE)---that
enabled the design and implementation of a coordination protocol for their
integration. The main contribution is to show how systems developed by two
independent teams can easily be integrated to provide new functionalities.
This suggests that end-to-end models of middleware, where a single ecosystem
of modules is designed to provide all the functionalities required to enable
distributed computing can be replaced by a model in which systems are
independently designed to be integrated. As seen in this paper, designing for
integration means developing a well-defined API, making explicit entities and
state models. Entities and functionalities must be chosen at the right level
of abstraction so to avoid the specificity of single implementations. For
example, NGE exposes Task, Core and Walltime entities. These are sufficiently
abstracted to be consistent with both systems’ design but specific enough to
define the domain of workload management for HPC.

Currently, we deployed the integrated system at OLCF on the OpenShift
platform and we have started testing and performance characterization. We
measured no appreciable overhead when executing a workload via NGE or RP's
native API. We have now to compare the execution of a standard ATLAS Geant4
workload with Harvester to the same execution performed with Harverser and
NGE. This comparison will measure integration overheads but also workload
execution performance. We expect that the execution on pilot will introduce
optimizations, especially related to AthenaMP~\cite{Crooks_2012} setup and
the total number of events processed per walltime unit. After this
characterization, we will be ready to concurrently execute heterogeneous
workloads, using the same pilot to run tasks of distinct users from diverse
domains. This will make PanDA a general purpose workload manager and will
open the possibility to explore new resource utilization patterns on
leadership-class machines.

In this context, PanDA will leverage the design of NGE and RADICAL-Pilot to
gain access to Summit, the new leadership-class machine managed by OLCF at
ORNL. Thanks to its design, RADICAL-Pilot has been ported to Summit in less
than two months and because of the isolation between workload management and
resource acquisition, no modification of Harvester is required to execute its
workloads on Summit. As a matter of fact, Harvester will be able to execute
general purpose workloads on both Titan and Summit, at the same time and
without any modification to its code.

Following this period of testing, characterization and adoption of a new
machine, the system integrating Harvester and NGE will be evaluated for
production deployment.


% ---------------------------------------------------------------------------
% Bibliography
% ---------------------------------------------------------------------------
\bibliography{main}

\end{document}
