%%%%%%%%%%%%%%%%%%%%%%% file template.tex %%%%%%%%%%%%%%%%%%%%%%%%%
%
% This is a template file for Web of Conferences Journal
%
% Copy it to a new file with a new name and use it as the basis
% for your article
%
%%%%%%%%%%%%%%%%%%%%%%%%%% EDP Science %%%%%%%%%%%%%%%%%%%%%%%%%%%%
%
%%%\documentclass[option]{webofc}
%%% "twocolumn" for typesetting an article in two columns format (default one column)
%
\documentclass{webofc}
\usepackage[varg]{txfonts}   % Web of Conferences font
%
% Put here some packages required or/and some personnal commands
%
%
\begin{document}
%
\title{PanDA and RADICAL-Pilot Integration: Enabling the Pilot Paradigm on HPC Resources}
%
% subtitle is optionnal
%
%%%\subtitle{Do you have a subtitle?\\ If so, write it here}

\author{\firstname{Pavlo} \lastname{Svirin}\inst{1}\fnsep\thanks{\email{Mail address for first
    author}} \and
        \firstname{Matteo} \lastname{Turilli}\inst{2}\fnsep\thanks{\email{Mail address for second
             author if necessary}} \and
        \firstname{Andre} \lastname{Merzky}\inst{2}\fnsep\thanks{\email{Mail address for last
             author if necessary}}
        % etc.
}

\institute{Brookhaven National Laboratory 
\and
           Rutgers University 
%\and
%           Last address
          }

\abstract{%
  Insert your english abstract here.
}
%
\maketitle
%
\section{Introduction}
\label{intro}
PanDA is the Workload Management System (WMS) used by the ATLAS experiment at the LHC to execute scientific applications on widely distributed resources.  PanDA is designed to support the execution of distributed workloads and workflows via pilots~\cite{turilli2017comprehensive}. Pilot-capable WMS enable high throughput execution of tasks via multi-level scheduling while supporting interoperability across multiple sites. This is particularly relevant for LHC experiments, where millions of tasks are executed across multiple sites every month, analyzing and producing petabytes of data.

The implementation of PanDA WMS consists of several interconnected subsystems, communicating via dedicated API or HTTP messaging, and implemented by one or more modules. Databases are used to store stateful entities like tasks, jobs and input/output data, and to store information about sites, resources, logs, and accounting. PanDA can execute ATLAS tasks on a variety of infrastructures, including high performance computing facilities. In the past three years, PanDA was used to execute … milion jobs on Titan, a leadership-class machine managed by the Oak Ridge Leadership Computing Facility (OLCF) at the Oak Ridge National Laboratory (ORNL). 

Ref.~\cite{panda-paper} describes in detail how PanDA has been deployed on Titan. Importantly, it shows that while PanDA implements the pilot abstraction to execute tasks on grid resources, on HPC facilities it submits job directly to the batch system of the machine. The size and duration of this job depends on resource availability and workload requirements. Resource availability is polled from Titan every ten minutes via the backfill functionality of the Moab scheduler and a job is submitted requiring the available amount of resources. The workload of this job is sized so to allow its execution within the available walltime. PanDA uses Titan’s resources to execute Geant4 event simulations, a specific type of workload that is amenable to execute via the machine’s batch system. 

In this paper, we present the integration between Harvester and Next Generation Executor (NGE). Harvester is a new resource-facing service developed by ATLAS to enable execution of ATLAS workloads on HPC machines, bringing further coherence in the PanDA stack across the support of different types of infrastructures, like grid and HPC. NGE is a REST interface developed for RADICAL-Pilot, a pilot system designed to support high-throughput computing on HPC infrastructures, including leadership-class machines like Titan. Different from Harvester, NGE enables to submit a pilot job via the batch system of Titan and then directly scheduling tasks on the acquired resources without queuing in the machine batch system. In this way, tasks can be executed immediately while respecting the policies of the HPC machine.

PanDA can benefit from pilot capabilities in several ways. Pilot do not require to package a workload into a batch queue submission script, simplifying the deployment requirement. Further, pilots enable the concurrent and sequential execution of a number of tasks, until the available walltime is exhausted. Concurrent because a pilot holds multiple nodes of an HPC machine enabling the execution of multiple tasks at the same time; sequential because when a task completes, another task can be executed on the freed resources. In this way, tasks can be late bound to an active pilot, depending on the current and remaining availability. This is important because, in principle, ATLAS would not have to assign a specific portion of tasks to an HPC machine but it could assign them only when the resources become available. Thus this would bring further coherence to ATLAS software stack, as pilots and late binding are already used for grid resources.

RADICAL-Pilot offers a number of distinctive features. Among these, the most relevant for ATLAS is the possibility to run arbitrary tasks on any given pilot. Specifically, RADICAL-Pilot decouples the management of a pilot, the coordination of task executions on that pilot, and what each task executes. For example, if a task executes a Geant4 simulation and another a molecular dynamic simulation, both tasks can run at the same time on the same pilot. Further, RADICAL-Pilot enables also the concurrent execution of different tasks on CPU and GPU, allowing for the full utilization of HPC worker nodes resources. This capabilities will enable PanDA to transition from a workload management system designed specifically to support the execution of ATLAS workloads, to a system for the execution of general purpose workloads on HPC machines.

\section{Harvester}

Harvester [Harvester] is a resource-facing service developed for ATLAS experiment at CERN since 2016 with a wide collaboration of experts.  It is stateless with a modular design to work with different resource types and workflows. The main objectives of Harvester are as follows: first, it should be a common machinery for pilot provisioning on all ATLAS computing resources on Grid. Second, it should provide a commonality layer bringing coherence to HPC implementations. Third, it should add a capability to timely optimize CPU allocation among various resource types to remove batch-level static partitioning. Finally, it should integrate the PanDA system and resources more tightly for new advanced workflows.

Figure 1. Harvester architecture

Figure 1 shows a schematic view of the Harvester architecture. Harvester is a stateless service with a local master database and a central slave database. The local database is used for real-time bookkeeping close to resources, and the central database is periodically synchronized with the local database to provide the resource information to the PanDA server. The PanDA server uses the information together with global overview of workload distribution in order to orchestrate behaviour of Harvester instances. Therefore, communication between harvester and the PanDA server is bidirectional. Harvester fetches job descriptions that have to be executed from PanDA Server using communicator component, same component is used to report back the status of these jobs. Harvester accesses resources through different plugins (Figure 2) which have been developed by resource experts. When Harvester instances run on edge nodes of HPC centers they access compute nodes through local HPC batch systems using HPC submission plugins. Status of the batches is tracked using monitor plugin. Input and output data are transferred with various stager and preparator plugins which implement Rucio [RUCIO], Globus Online [Globus_online]  and other data transfer clients. All of the communication between the components is done via local database.



Figure. Modules used for PanDA-NGE interaction

In context of BigPanDA project several instances of Harvester have been deployed on front nodes of BNL Institutional Cluster, NERSC (Cori), Titan and Jefferson Lab clusters. They are serving experiments like ATLAS, LQCD, nEDM, IceCube and LSST. 


\section{RADICAL-Pilot and Next Generation Executor}

RADICAL-Pilot (RP) is a pilot system~\cite{turillirp} which enables task-level concurrency on a variety of infrastructures, including XSEDE HPC, Cloud and grid systems, NARC Cheyenne, NCSA Blue Waters, and ORNL Reha, Titan and Summit~\cite{rp-journal}. As a pilot system, RP allows scheduling jobs on HPC machines (or virtual machines on cloud infrastructures) to acquire computing resources. Once these resources become available, RP schedules tasks on them directly, i.e., without using the machine’s batch system. In this way, tasks do not wait in the queue but execute immediately enabling high-throughput on HPC machines. It is important to stress that RP, and pilot systems in general, do not “game” the scheduling policies of the HPC machines: resource acquisition is performed via the batch system and resources remain available only for the given walltime period. Jobs waits in the queue alongside all the other jobs and are subjected to queue, allocation and fair-use policies of the system.
	
	Among pilot systems, RP has unique capabilities and architectural properties. RP offers concurrent execution of heterogeneous tasks on the same pilot, supporting both CPU and GPU, more than twelve task launching methods---e.g., ssh, mpirun, aprun, openmpi---and all the major HPC batch systems---e.g., SLURM, PBS, Torque. RP can execute single or multi core tasks within a single compute node, or across multiple nodes. RP isolates the execution of each tasks into a dedicated process, enabling concurrent execution of heterogeneous tasks by design. For example, RP can run a bag of 65,000 heterogeneous tasks, requiring between 1 and 384 cores, some running on CPU cores, others on GPU, some using OpenMP other MPI. These tasks may run in several ‘generations’, each with varying degree of concurrency and combination of types of tasks. Further RP can concurrently manage multiple pilots both on the same machine or submitted across a set of machines. Tasks can be late-bound to available resources, using different scheduling algorithms across resources and within each resource.
	
	Architecturally, RP is designed following a so called ‘building blocks’ design approach~\cite{bb}. Consistently, RP is self-sufficient, interoperable, extensible and partially composable. Self-sufficient because RP independently implements the necessary and sufficient set of functionalities for describing and managing pilot and task entities; interoperable in terms of type of workload, resource, and execution paradigm; and extensible as new properties can be added to the pilot, task and resource descriptions, and more functionalities can be implemented for this entities. Currently, composability is partially designed and implemented: while the pilot-API can be used by both users and other systems to describe one or more task-based workloads for execution, RADICAL-Pilot requires RADICAL-SAGA to interface to HPC resources. A prototype interface to cloud resources based on LibCloud is available and a general-purpose resource connector component is under development.
	
	
	Figure #. RADICAL-Pilot (RP) architecture and execution model. Architecture: interfaces (white), modules (purple), components (yellow), pilot entity (green), task entity called compute unit (red). Execution model: Pilot and units are described via the Pilot API (1); pilots are submitted to the indicated resource(s) (2); each pilot bootstraps an agent (3); compute units are scheduled onto available agent(s) (4) and executed (5).
	
	Figure~\ref{fig:rp-arch} shows RP architecture and execution model. RP assumes a multi-task application written in Python, using the Pilot API. This API let users to describe tasks---called compute units (CU)---and pilots (Figure~\ref{fig:rp-arch}.1). Each task description has a set of properties, including the name of the executable launched by RP, the type and amount of cores this executable will need, whether it requires MPI, input files and so on. Pilot descriptions also have a set of properties, including the endpoint where the resource request should be submitted, the type and amount of resources to acquire, and the walltime of these resources. The RP API includes classes to create both pilot and unit managers to which pilot and unit descriptions are assigned for execution. Once managed, pilots are submitted to the indicated endpoint (Figure~\ref{fig:rp-arch}.2), and once scheduled pilot bootstrap an agent (Figure~\ref{fig:rp-arch}.3). Unit manager(s) schedule compute units onto available agent(s) (Figure~\ref{fig:rp-arch}.4) and each agent uses a resource-specific executor to run the units (Figure~\ref{fig:rp-arch}.5). 
	
	Note that compute units are scheduled once the agent is active and that unit descriptions need to be transferred from the client to the agent. This means that the latency of each unit transfer reduces the overall utilization of the available resources as no unit can be executed while in transit. To address this issue, RP transfers units in bulk, reducing the overall latency of unit transfer to a single round trip. Further, units are scheduled via a database instance that needs to be reachable by both the machine on which the unit managers is instantiated, and the remote machine on which the units will have to be executed. 
	
	RP supports the execution of units which require staging of input and/or output data. Units’ input and output files are managed by stager components not represented in Figure~\ref{fig:rp-arch}. Data staging is performed independently of the pilot lifetime, and can thus overlap with the pilot's queue waiting time, thus reducing the impact on resource utilization.
	
	The next generation executor (NGE) is a REST API that enables running RP as a service. The REST API is a close semantic representation of the underlying native RP API.  The service manages the lifetime of the RP unit and pilot managers, and of the RP pilot agent, on behalf of a client. As is often the case when translating a module or library API (which is usually invoked in a process-local call stack) into a REST API (which is usually called over a network link), care has to be taken to not let even moderate network latencies impact the overall API performance.  The NGE API manages to avoid that problem by adding support for bulk operations for those API methods most prone to latency impacts: unit submission and state updates.
	
	NGE service also provides functionality in addition to the RP API scope: instead of requiring the client to specify pilot size, runtime and configuration, it implements policy driven automatisms to shape a pilot based on the received workload and on resource availability.  The service will submit those auto-shaped pilots to the resource batch queue -- but it is also able to inspect the target resource's backfill availability, and to shape the pilots specifically so that they fit the available backfill constraints. 
	
	NGE's pilot shaping capabilities are policy driven: they can be adapted to the configuration of the target resource, but also toward specific groups of use cases. This makes the NGE deployments configurable for specific user groups and projects.  In the context of the presented use case, that configurability is used to mimic properties which were available in Panda's native execution backends, which eases integration semantically.


\section{Integrating Harvester and NGE}

Figure #. Integration between Harvester and NGE deployed at OLCF to manage the execution of Particle Physics and Molecular Dynamics workloads. Harvester, NGE and RP, and MongoDB are deployed on containers provided and managed via OpenShift.

We deployed Harvester, NGE and RP, and the MongoDB instance needed by RP on three separate containers provided by OLCF via their OpenShift service. The container used for NGE and RP can directly submit jobs to the PBS batch system of Titan, while the MongoDB instance can be reached by both Titan and the NGE and RP container. This enable submission of pilot jobs to Titan and scheduling of tasks on the resources of that pilot once scheduled and bootstrapped.

A separate instance of Harvester has been set up in order to run experiments of PanDA and NGE interaction (Fig. 3) on a front node of Titan supercomputer. Because of differences in environments Harvester and NGE instances have been installed on different nodes (data transfer node for Harvester and login node for NGE), the communication is done via a secure tunnel between these nodes. A special module for job submissions to NGE and job monitoring has been developed. The functionality of these modules have been tested with dummy jobs as well as with samples of ATLAS and Molecular Dynamics [MD]. Another NGE instance that will be running in a container in OpenShift at OLCF is under active deployment.

Figure~\ref{fig:integration} shows details of the coordination protocol between Harvester and NGE and of the execution model of the integrated system. RP initiates resource acquisition by submitting a configurable number of pilots to Titan (Figure~\ref{fig:integration}.1). Once one or more pilots become available, RP publishes the aggregate number of cores and their time availability via NGE (Figure~\ref{fig:integration}.2). Note how NGE abstracts the granularity of pilot resources into a resource overlay described by the total core availability, partitioned based on the amount of time cores are available. This partitioning is made necessary by the stacked availability of multiple pilots on the same resource. 

Harvester can pull NGE to know how many cores are available and for how long (Figure~\ref{fig:integration}.3) and push a number of tasks to NGE for execution (Figure~\ref{fig:integration}.5). Note that before pushing tasks to NGE, Harvester makes sure that the input data required by these tasks are available to the work nodes of Titan. This is done by copying or linking the relevant data to a shared file system, either local to the cluster or available on the OLCF network (Figure~\ref{fig:integration}.5).  The number of tasks pushed to NGE is calculate on the base of aggregated task requirements in terms of number of cores required and walltime. Tasks are pushed in bulk so to avoid latency and other overheads associated with pushing single tasks. Once tasks are pushed into NGE, these are translated into tasks descriptions for RP (Figure~\ref{fig:integration}.6) and then executed on the available pilot resources are described in Sec.~\ref{sec:rp} (Figure~\ref{fig:integration}.7-9). Once executed, RP stages out tasks’ output to a filesystem accessible by Harvester (Figure~\ref{fig:integration}.10) and Harvester collects this files, terminating the execution cycle (Figure~\ref{fig:integration}.11).

NGE can be configured to use RP to submit jobs to Titan in two modes: standard or backfill. Standard mode supports submission of jobs to one or more of the five PBS queues available on Titan. In this mode, job submissions are exactly the same as any other job submitted to Titan for execution by any other user. In backfill mode, RP uses the … command of the Maui scheduler to poll the current node and walltime availability, creates a pilot description requesting that availability and submits an equivalent job to the required queue. In this way, RP increases the chances for that job to be immediately scheduled on Titan, reducing the job (and therefore pilot) queue waiting time. In this way, NGE can operate as Harvester but offering full-pilot capabilities. This allows for Harvester to submit multiple generations of tasks for execution on Titan, maintaining a predefined ‘pressure’ on Titan queues. This makes the Havester/NGE integrated system execution model and operational modality consistent across Grid and HPC, reducing the differences in how tasks are mapped between the two types of infrastructure.

\section{Conclusions}

In this paper we introduced two distinct systems and an interface that enabled the design and implementation of a coordination protocol for their integration. The main contribution is to show how systems developed by two independent teams can easily be integrated to provide new functionalities. This suggests that end-to-end models of middleware where a single ecosystem of modules is design to provide all the functionalities required to enable distributed computing can be replaced by a model in which systems are designed to be integrated. As seen in this paper, designing for integration means developing a well-define API, making explicit the entities that can be defined via this API and the functionalities upon these entities. Entities and functionalities must be chosen at the right level of abstraction so to avoid the specificity of single implementations. For example, NGE exposes Task, Core and Walltime entities. These are sufficiently abstracted to be consistent with both systems’ design but specific enough to define the domain of workload management for HPC.

Currently, we deployed the integrated system at OLCF on the OpenShift platform and we have started testing and performance characterization. As seen in Sec. 5, the integration between NGE and RADICAL-Pilot has created no appreciable overheads. We have now to compare the execution of a standard ATLAS Geant4 workload with Harvester to the same execution performed with Harverser and NGE. This comparison will measure integration overheads but also workload execution performance. We expect that the execution on pilot will introduce optimizations, especially related to AthenaMP setup and the total number of events processed per walltime unit. After this characterization, we will be ready to concurrently execute heterogeneous workloads, using the same pilot to run tasks of distinct users from diverse domains. This will make PanDA a general purpose workload manager and will open the possibility to explore new resource utilization patterns on leadership-class machines. 

In this context, PanDA will leverage the design of NGE and RADICAL-Pilot to gain access to Summit, the new leadership-class machine managed by OLCF at ORNL. Thanks to its design, RADICAL-Pilot has been ported to Summit in less than two months and because of the isolation between workload management and resource acquisition, no modification of Harvester is required to execute its workloads on Summit. As a matter of fact, Harvester will be able to execute general purpose workloads on both Titan and Summit, at the same time and without any modification to its code.

Following this period of testing, characterization and adoption of a new machine, the system integrating Harvester and NGE will be evaluated for production deployment.




\section{Section title}
\label{sec-1}
For bibliography use \cite{RefJ}

\subsection{Subsection title}
\label{sec-2}
Don't forget to give each section, subsection, subsubsection, and
paragraph a unique label (see Sect.~\ref{sec-1}).

For one-column wide figures use syntax of figure~\ref{fig-1}
\begin{figure}[h]
% Use the relevant command for your figure-insertion program
% to insert the figure file.
\centering
%\includegraphics[width=1cm,clip]{tiger}
\caption{Please write your figure caption here}
\label{fig-1}       % Give a unique label
\end{figure}

For two-column wide figures use syntax of figure~\ref{fig-2}
\begin{figure*}
\centering
% Use the relevant command for your figure-insertion program
% to insert the figure file. See example above.
% If not, use
\vspace*{5cm}       % Give the correct figure height in cm
\caption{Please write your figure caption here}
\label{fig-2}       % Give a unique label
\end{figure*}

For figure with sidecaption legend use syntax of figure
\begin{figure}
% Use the relevant command for your figure-insertion program
% to insert the figure file.
\centering
\sidecaption
%\includegraphics[width=5cm,clip]{tiger}
\caption{Please write your figure caption here}
\label{fig-3}       % Give a unique label
\end{figure}

For tables use syntax in table~\ref{tab-1}.
\begin{table}
\centering
\caption{Please write your table caption here}
\label{tab-1}       % Give a unique label
% For LaTeX tables you can use
\begin{tabular}{lll}
\hline
first & second & third  \\\hline
number & number & number \\
number & number & number \\\hline
\end{tabular}
% Or use
\vspace*{5cm}  % with the correct table height
\end{table}
%
% BibTeX or Biber users please use (the style is already called in the class, ensure that the "woc.bst" style is in your local directory)
% \bibliography{name or your bibliography database}
%
% Non-BibTeX users please use
%
\begin{thebibliography}{}
%
% and use \bibitem to create references.
%
\bibitem{RefJ}
% Format for Journal Reference
Journal Author, Journal \textbf{Volume}, page numbers (year)
% Format for books
\bibitem{RefB}
Book Author, \textit{Book title} (Publisher, place, year) page numbers
% etc
\end{thebibliography}

\end{document}

% end of file template.tex

<div id='footer'><table width='100%'><tr><td class='right'><a href='http://fusioninventory.org/'><span class='copyright'>FusionInventory 9.1+1.0 | copyleft <img src='/glpi/plugins/fusioninventory/pics/copyleft.png'/>  2010-2016 by FusionInventory Team</span></a></td></tr></table></div>